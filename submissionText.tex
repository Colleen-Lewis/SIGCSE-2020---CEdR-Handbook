%%%% Proceedings format for most of ACM conferences (with the exceptions listed below) and all ICPS volumes.
\documentclass[sigconf]{acmart}

% defining the \BibTeX command - from Oren Patashnik's original BibTeX documentation.
\def\BibTeX{{\rm B\kern-.05em{\sc i\kern-.025em b}\kern-.08emT\kern-.1667em\lower.7ex\hbox{E}\kern-.125emX}}
\usepackage{makecell}
    
% Rights management information. 
\copyrightyear{2020}
\acmYear{2020}
\setcopyright{acmlicensed}
\acmConference[SIGCSE '20]{SIGCSE '20: Special Interest Group for Computer Science Education}{March 11--14, 2020}{Portland, Oregon, USA}
\acmPrice{15.00}
\acmDOI{10.1145/1122445.1122456}
\acmISBN{978-1-4503-9999-9/18/06}

%
% These commands are for a JOURNAL article.
%\setcopyright{acmcopyright}
%\acmJournal{TOG}
%\acmYear{2018}\acmVolume{37}\acmNumber{4}\acmArticle{111}\acmMonth{8}
%\acmDOI{10.1145/1122445.1122456}

%
% Submission ID. 
% Use this when submitting an article to a sponsored event. You'll receive a unique submission ID from the organizers
% of the event, and this ID should be used as the parameter to this command.
%\acmSubmissionID{123-A56-BU3}

%
% The majority of ACM publications use numbered citations and references. If you are preparing content for an event
% sponsored by ACM SIGGRAPH, you must use the "author year" style of citations and references. Uncommenting
% the next command will enable that style.
%\citestyle{acmauthoryear}

%
% end of the preamble, start of the body of the document source.
\begin{document}
\fancyhead{} 

%
% The "title" command has an optional parameter, allowing the author to define a "short title" to be used in page headers.
\title[The Cambridge Handbook in 75 minutes]{The Cambridge Handbook of Computing Education Research Summarized in 75 minutes}

\author{Colleen M. Lewis (Emcee)}
\affiliation{
  \institution{Harvey Mudd College}
  \city{Claremont}
  \state{CA}
  \country{USA}
  \postcode{91711}
}
\email{lewis@cs.hmc.edu}

\author{Paulo Blikstein}
\affiliation{
  \institution{Stanford University}
  \city{Stanford}
  \state{CA}
  \country{USA}
  \postcode{94305}
}
\email{paulob@stanford.edu}


\author{Katrina Falkner}
\affiliation{
  \institution{University of Adelaide}
  \city{Adelaide}
  \state{SA, Australia}
  \postcode{5005}
}
\email{katrina.falkner@adelaide.edu.au}

\author{Sally A. Fincher}
\affiliation{
  \institution{University of Kent}
  \city{Canterbury}
  \state{Kent}
  \country{UK}
  \postcode{CT2 7NF}
}
\email{s.a.fincher@kent.ac.uk}

\author{Kathi Fisler}
\affiliation{
  \institution{Brown University}
  \city{Providence}
  \state{RI}
  \country{USA}
  \postcode{02912}
}
\email{kfisler@cs.brown.edu}

\author{Mark Guzdial}
\affiliation{
  \institution{University of Michigan}
  \city{Ann Arbor}
  \state{MI}
  \country{USA}
  \postcode{48103}
}
\email{mjguz@umich.edu}

\author{Patricia Haden}
\affiliation{
  \institution{University of Otago, New Zealand}
  \city{Dunedin}
  \country{New Zealand}
  \postcode{9054}
}
\email{patriciahaden1@gmail.com}


\author{Christopher Hundhausen}
\affiliation{
  \institution{Washington State University}
  \city{Pullman}
  \state{WA}
  \country{USA}
  \postcode{99164}
}
\email{hundhaus@wsu.edu}

\author{Amy J. Ko}
\affiliation{
  \institution{University of Washington, Seattle}
  \city{Seattle}
  \state{WA}
  \country{USA}
  \postcode{98195}
}
\email{ajko@uw.edu}

\author{Thomas Lancaster}
\affiliation{
  \institution{Imperial College London}
  \city{London}
  \country{UK}
  \postcode{SW7 2RH}
}
\email{thomas@thomaslancaster.co.uk}


\author{Michael C. Loui}
\affiliation{
  \institution{Purdue University}
  \city{West Lafayette}
  \state{IN}
  \country{USA}
  \postcode{47907}
}
\email{mloui@purdue.edu}


\author{Lauren Margulieux}
\affiliation{
  \institution{Georgia State University}
  \city{Atlanta}
  \state{GA}
  \country{USA}
  \postcode{30302}
}
\email{lmargulieux@gsu.edu}



\author{Leo Porter}
\affiliation{
  \institution{University of California, San Diego}
  \city{La Jolla}
  \state{CA}
  \country{USA}
  \postcode{92093}
}
\email{leporter@cs.ucsd.edu}

\author{Anthony Robins}
\affiliation{
  \institution{University of Otago, New Zealand}
  \city{Dunedin}
  \country{New Zealand}
  \postcode{9054}
}
\email{anthony@cs.otago.ac.nz}


\author{Niral Shah}
\affiliation{
  \institution{University of Washington}
  \city{Seattle}
  \state{WA}
  \country{USA}
  \postcode{98195}
}
\email{niral@uw.edu}

\author{R. Benjamin Shapiro}
\affiliation{
  \institution{University of Colorado Boulder}
  \city{Boulder}
  \state{CO}
  \country{USA}
  \postcode{80309}
}
\email{ben.shapiro@colorado.edu}

\author{Kerry Shephard}
\affiliation{
  \institution{University of Otago, New Zealand}
  \city{Dunedin}
  \country{New Zealand}
  \postcode{9054}
}
\email{kerry.shephard@otago.ac.nz}

\author{Mike Tissenbaum}
\affiliation{
  \institution{University of Illinois at Urbana Champaign}
  \city{Champaign}
  \state{IL}
  \country{USA}
  \postcode{61820}
}
\email{miketissenbaum@gmail.com}

\author{Ian Utting}
\affiliation{
  \institution{University of Kent}
  \city{Canterbury}
  \state{Kent}
  \country{UK}
}
\email{I.A.Utting@kent.ac.uk}

\author{Jan Vahrenhold}
\affiliation{
  \institution{Westf{\"a}lische Wilhelms-Universit{\"at} M{\"u}nster}
  \city{M{\"u}nster}
  \country{Germany}
  \postcode{48149}
}
\email{jan.vahrenhold@uni-muenster.de}




\author{Aman Yadav}
\affiliation{
  \institution{Michigan State University}
  \city{East Lansing}
  \state{MI}
  \postcode{48824}
  \country{USA}
}
\email{ayadav@msu.edu}


%
% By default, the full list of authors will be used in the page headers. Often, this list is too long, and will overlap
% other information printed in the page headers. This command allows the author to define a more concise list
% of authors' names for this purpose.
\renewcommand{\shortauthors}{Lewis, et al.}

%
% The abstract is a short summary of the work to be presented in the article.
\begin{abstract}
The 32 chapters of the 2019 Cambridge Handbook of Computing Education Research synthesize the existing research in computing education and propose new directions for future research. An author from each chapter will summarize their chapter with auto-advancing slides. Attendees will be introduced to the breadth of content in the new handbook and can identify chapters of interest. This fits uniquely as a special session, and will likely be informative, inspiring, and overwhelming. 
\end{abstract}

\begin{CCSXML}
<ccs2012>
<concept>
<concept_id>10003456.10003457.10003527.10003528</concept_id>
<concept_desc>Social and professional topics~Computational thinking</concept_desc>
<concept_significance>500</concept_significance>
</concept>
<concept>
<concept_id>10003456.10003457.10003527.10003531.10003533</concept_id>
<concept_desc>Social and professional topics~Computer science education</concept_desc>
<concept_significance>500</concept_significance>
</concept>
<concept>
<concept_id>10003456.10003457.10003527.10003541</concept_id>
<concept_desc>Social and professional topics~K-12 education</concept_desc>
<concept_significance>500</concept_significance>
</concept>
</ccs2012>
\end{CCSXML}


%
% Keywords. 
\keywords{Computing education research, literature review, methods}
    
\maketitle

\section{Session Details} 
Each chapter \cite{ch00, ch01, ch02, ch03, ch04, ch05, ch06, ch08, ch09, ch10, ch11, ch12, ch13, ch14, ch15, ch16, ch17, ch18, ch19, ch20, ch21, ch22, ch23, ch24, ch25, ch26, ch27, ch28, ch29, ch30, ch31}  will be presented with a timed, 105 second talk and followed by a 30-second, transition (72 minutes). Twitter handles will be advertised for each talk and attendees will be encouraged to post questions on Twitter or the Whova app.  

\begin{comment}
\begin{acks}
Acknowledgement text.
\end{acks}
\end{comment}
%
% The next two lines define the bibliography style to be used, and the bibliography file.
\bibliographystyle{ACM-Reference-Format}
\bibliography{sample-base}

\end{document}
