%%%% Proceedings format for most of ACM conferences (with the exceptions listed below) and all ICPS volumes.
\documentclass[sigconf]{acmart}

% defining the \BibTeX command - from Oren Patashnik's original BibTeX documentation.
\def\BibTeX{{\rm B\kern-.05em{\sc i\kern-.025em b}\kern-.08emT\kern-.1667em\lower.7ex\hbox{E}\kern-.125emX}}
\usepackage{makecell}
    
% Rights management information. 
\copyrightyear{2020}
\acmYear{2020}
\setcopyright{acmlicensed}
\acmConference[SIGCSE '20]{SIGCSE '20: Special Interest Group for Computer Science Education}{March 11--14, 2020}{Portland, Oregon, USA}
\acmPrice{15.00}
\acmDOI{10.1145/1122445.1122456}
\acmISBN{978-1-4503-9999-9/18/06}

%
% These commands are for a JOURNAL article.
%\setcopyright{acmcopyright}
%\acmJournal{TOG}
%\acmYear{2018}\acmVolume{37}\acmNumber{4}\acmArticle{111}\acmMonth{8}
%\acmDOI{10.1145/1122445.1122456}

%
% Submission ID. 
% Use this when submitting an article to a sponsored event. You'll receive a unique submission ID from the organizers
% of the event, and this ID should be used as the parameter to this command.
%\acmSubmissionID{123-A56-BU3}

%
% The majority of ACM publications use numbered citations and references. If you are preparing content for an event
% sponsored by ACM SIGGRAPH, you must use the "author year" style of citations and references. Uncommenting
% the next command will enable that style.
%\citestyle{acmauthoryear}

%
% end of the preamble, start of the body of the document source.
\begin{document}

%
% The "title" command has an optional parameter, allowing the author to define a "short title" to be used in page headers.
\title[The Cambridge Handbook in 75 minutes]{The Cambridge Handbook of Computing Education Research Summarized in 75 minutes}

\author{Author Name}
\affiliation{%
  \institution{Institution name}
  \streetaddress{Institution address}
  \city{institution city}
  \state{State}
  \postcode{99999}
}
\email{Author Email}

\author{Author Name}
\affiliation{%
  \institution{Institution name}
  \streetaddress{Institution address}
  \city{institution city}
  \state{State}
  \postcode{99999}
}

\email{Author Email}
\author{Author Name}
\affiliation{%
  \institution{Institution name}
  \streetaddress{Institution address}
  \city{institution city}
  \state{State}
  \postcode{99999}
}

\email{Author Email}
\author{Author Name}
\affiliation{%
  \institution{Institution name}
  \streetaddress{Institution address}
  \city{institution city}
  \state{State}
  \postcode{99999}
}
\email{Author Email}

\author{Author Name}
\affiliation{%
  \institution{Institution name}
  \streetaddress{Institution address}
  \city{institution city}
  \state{State}
  \postcode{99999}
}
\email{Author Email}

\author{Author Name}
\affiliation{%
  \institution{Institution name}
  \streetaddress{Institution address}
  \city{institution city}
  \state{State}
  \postcode{99999}
}
\email{Author Email}

\author{Author Name}
\affiliation{%
  \institution{Institution name}
  \streetaddress{Institution address}
  \city{institution city}
  \state{State}
  \postcode{99999}
}
\email{Author Email}

\author{Author Name}
\affiliation{%
  \institution{Institution name}
  \streetaddress{Institution address}
  \city{institution city}
  \state{State}
  \postcode{99999}
}

\email{Author Email}
\author{Author Name}
\affiliation{%
  \institution{Institution name}
  \streetaddress{Institution address}
  \city{institution city}
  \state{State}
  \postcode{99999}
}

\email{Author Email}
\author{Author Name}
\affiliation{%
  \institution{Institution name}
  \streetaddress{Institution address}
  \city{institution city}
  \state{State}
  \postcode{99999}
}
\email{Author Email}

\author{Author Name}
\affiliation{%
  \institution{Institution name}
  \streetaddress{Institution address}
  \city{institution city}
  \state{State}
  \postcode{99999}
}
\email{Author Email}

\author{Author Name}
\affiliation{%
  \institution{Institution name}
  \streetaddress{Institution address}
  \city{institution city}
  \state{State}
  \postcode{99999}
}
\email{Author Email}

\author{Author Name}
\affiliation{%
  \institution{Institution name}
  \streetaddress{Institution address}
  \city{institution city}
  \state{State}
  \postcode{99999}
}
\email{Author Email}

\author{Author Name}
\affiliation{%
  \institution{Institution name}
  \streetaddress{Institution address}
  \city{institution city}
  \state{State}
  \postcode{99999}
}

\email{Author Email}
\author{Author Name}
\affiliation{%
  \institution{Institution name}
  \streetaddress{Institution address}
  \city{institution city}
  \state{State}
  \postcode{99999}
}

\email{Author Email}

\author{Author Name}
\affiliation{%
  \institution{Institution name}
  \streetaddress{Institution address}
  \city{institution city}
  \state{State}
  \postcode{99999}
}
\email{Author Email}

\author{Author Name}
\affiliation{%
  \institution{Institution name}
  \streetaddress{Institution address}
  \city{institution city}
  \state{State}
  \postcode{99999}
}
\email{Author Email}

\author{Author Name}
\affiliation{%
  \institution{Institution name}
  \streetaddress{Institution address}
  \city{institution city}
  \state{State}
  \postcode{99999}
}
\email{Author Email}

\author{Author Name}
\affiliation{%
  \institution{Institution name}
  \streetaddress{Institution address}
  \city{institution city}
  \state{State}
  \postcode{99999}
}
\email{Author Email}

\author{Author Name}
\affiliation{%
  \institution{Institution name}
  \streetaddress{Institution address}
  \city{institution city}
  \state{State}
  \postcode{99999}
}
\email{Author Email}

\author{Author Name}
\affiliation{%
  \institution{Institution name}
  \streetaddress{Institution address}
  \city{institution city}
  \state{State}
  \postcode{99999}
}
\email{Author Email}

\author{Author Name}
\affiliation{%
  \institution{Institution name}
  \streetaddress{Institution address}
  \city{institution city}
  \state{State}
  \postcode{99999}
}
\email{Author Email}


\author{Author Name}
\affiliation{%
  \institution{Institution name}
  \streetaddress{Institution address}
  \city{institution city}
  \state{State}
  \postcode{99999}
}
\email{Author Email}

\author{Author Name}
\affiliation{%
  \institution{Institution name}
  \streetaddress{Institution address}
  \city{institution city}
  \state{State}
  \postcode{99999}
}
\email{Author Email}
%
% By default, the full list of authors will be used in the page headers. Often, this list is too long, and will overlap
% other information printed in the page headers. This command allows the author to define a more concise list
% of authors' names for this purpose.
\renewcommand{\shortauthors}{Authors, et al.}

%
% The abstract is a short summary of the work to be presented in the article.
\begin{abstract}
The Cambridge Handbook of Computing Education Research was published in 2019, and the 32 chapters synthesize the existing research in computing education and proposes new directions. In this special session, an author from each chapter will summarize their chapter for the audience using only seven slides, which auto advance every 15 seconds. After a 30-second transition, we will move on to the next chapter! In this 75-minute special session, attendees will be introduced to the breadth of the content within the new handbook and hopefully identify chapters of interest. 
\end{abstract}

\begin{CCSXML}
<ccs2012>
<concept>
<concept_id>10003456.10003457.10003527.10003528</concept_id>
<concept_desc>Social and professional topics~Computational thinking</concept_desc>
<concept_significance>500</concept_significance>
</concept>
<concept>
<concept_id>10003456.10003457.10003527.10003531.10003533</concept_id>
<concept_desc>Social and professional topics~Computer science education</concept_desc>
<concept_significance>500</concept_significance>
</concept>
<concept>
<concept_id>10003456.10003457.10003527.10003541</concept_id>
<concept_desc>Social and professional topics~K-12 education</concept_desc>
<concept_significance>500</concept_significance>
</concept>
</ccs2012>
\end{CCSXML}


%
% Keywords. 
\keywords{Computing education research, literature review, methods}
    
\maketitle

\section{Introduction} 
The objective of the session is to introduce the breadth of the content within the 2019 The Cambridge Handbook of Computing Education Research and help attendees identify chapters of interest. The outline of the session is relatively simple; each chapter \cite{ch00, ch01, ch02, ch03, ch04, ch05, ch06, ch07, ch08, ch09, ch10, ch11, ch12, ch13, ch14, ch15, ch16, ch17, ch18, ch19, ch20, ch21, ch22, ch23, ch24, ch25, ch26, ch27, ch28, ch29, ch30, ch31}  will be presented during a 105 second lightning talk and followed by a 30-second, silent transition (72 minutes). Twitter handles will be advertised for each chapter and during these brief transitions, attendees will be encouraged to post questions about the chapter on Twitter or the Whova app. This creative format fits uniquely as a special session, and will likely be informative, inspiring, and overwhelming. 

\begin{comment}
\section{Chapter Presentations} 
The chapters will be presented in order of appearance as enumerated below. 
\begin{itemize}						
	\item "An important and timely field" presented by Sally A. Fincher \cite{ch00}
\end{itemize}						
\subsection{Part I Background}
\begin{itemize}						
	\item "The History of computing education research" presented by Mark Guzdial \cite{ch01}
	\item "Computing education research today" presented by Sally A. Fincher \cite{ch02}
	\item "Computing education: Literature review and voices from the field" presented by Paulo Blikstein \cite{ch03}
\end{itemize}	
\subsection{Part II Foundations}
\begin{itemize}						
	\item "A study design process" presented by Andrew J. Ko \cite{ch04}
	\item "Descriptive statistics" presented by Patricia Haden \cite{ch05}
	\item "Inferential Statistics" presented by Patricia Haden \cite{ch06}
	\item "Qualitative methods for computing education" presented by Josh Tenenberg \cite{ch07}
	\item "Learning sciences for computing education" presented by Lauren E. Margulieux \cite{ch08}
	\item "Cognitive sciences for computing education" presented by Anthony Robins \cite{ch09}
	\item "Higher education pedagogy" presented by Professor Kerry Shephard \cite{ch10}
	\item "Engineering education research" presented by Michael C. Loui \cite{ch11}
	\item "Novice programmers and introductory programming" presented by Anthony Robins \cite{ch12}
\end{itemize}						
\subsection{Part III Topics: Systemic Issues}
\begin{itemize}						
	\item "Programming paradigms and beyond" presented by Shriram Krishnamurthi \cite{ch13}
	\item "Assessment and plagiarism" presented by Thomas Lancaster \cite{ch14}
	\item "Pedagogic approaches" presented by Katrina Falkner \cite{ch15}
	\item "Equity and diversity" presented by Colleen M. Lewis \cite{ch16}
\end{itemize}
\subsection{Part III Topics: New Milieux}
\begin{itemize}						
	\item "Computational thinking" presented by Paul Curzon \cite{ch17}
	\item "Schools (K-12)" presented by Jan Vahrenhold \cite{ch18}
	\item "Computing for other disciplines" presented by Mark Guzdial \cite{ch19}
	\item "New programming paradigms" presented by R. Benjamin Shapiro \cite{ch20}
\end{itemize}						

\subsection{Part III Topics: Systems Software and Technology}

\begin{itemize}						
	\item "Tools and environments" presented by Lauri Malmi \cite{ch21}
	\item "Tangible computing" presented by Michael Horn \cite{ch22}
	\item "Leveraging the IDE for learning analytics" presented by Adam Carter \cite{ch23}
\end{itemize}						
\subsection{Part III Topics: Teacher and Student Knowledge}
\begin{itemize}						
	\item "Teacher knowledge for inclusive computing learning" presented by Joanna Goode \cite{ch24}
	\item "Teacher learning and development" presented by Sally. A. Fincher \cite{ch25}
	\item "Learning outside the classroom" presented by Andrew Begel \cite{ch26}
	\item "Student knowledge and misconceptions" presented by Colleen M. Lewis \cite{ch27}
	\item "Motivation, attitudes and dispositions" presented by Alex Lishinski \cite{ch28}
	\item "Students as teachers and communicators" presented by Beth Simon \cite{ch29}
\end{itemize}						
\subsection{Part III Topics: Case Studies}
\begin{itemize}						
	\item "A case study of peer instruction" presented by Leo Porter \cite{ch30}
	\item "A case study of qualitative methods" presented by Colleen M. Lewis \cite{ch31}
\end{itemize}						



\begin{acks}
Acknowledgement text.
\end{acks}
\end{comment}
%
% The next two lines define the bibliography style to be used, and the bibliography file.
\bibliographystyle{ACM-Reference-Format}
\bibliography{sample-base}

\end{document}
