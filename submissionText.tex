%%%% Proceedings format for most of ACM conferences (with the exceptions listed below) and all ICPS volumes.
\documentclass[sigconf]{acmart}

% defining the \BibTeX command - from Oren Patashnik's original BibTeX documentation.
\def\BibTeX{{\rm B\kern-.05em{\sc i\kern-.025em b}\kern-.08emT\kern-.1667em\lower.7ex\hbox{E}\kern-.125emX}}
\usepackage{makecell}
    
% Rights management information. 
\copyrightyear{2020}
\acmYear{2020}
\setcopyright{acmlicensed}
\acmConference[SIGCSE '20]{SIGCSE '20: Special Interest Group for Computer Science Education}{March 11--14, 2020}{Portland, Oregon, USA}
\acmPrice{15.00}
\acmDOI{10.1145/1122445.1122456}
\acmISBN{978-1-4503-9999-9/18/06}

%
% These commands are for a JOURNAL article.
%\setcopyright{acmcopyright}
%\acmJournal{TOG}
%\acmYear{2018}\acmVolume{37}\acmNumber{4}\acmArticle{111}\acmMonth{8}
%\acmDOI{10.1145/1122445.1122456}

%
% Submission ID. 
% Use this when submitting an article to a sponsored event. You'll receive a unique submission ID from the organizers
% of the event, and this ID should be used as the parameter to this command.
%\acmSubmissionID{123-A56-BU3}

%
% The majority of ACM publications use numbered citations and references. If you are preparing content for an event
% sponsored by ACM SIGGRAPH, you must use the "author year" style of citations and references. Uncommenting
% the next command will enable that style.
%\citestyle{acmauthoryear}

%
% end of the preamble, start of the body of the document source.
\begin{document}

%
% The "title" command has an optional parameter, allowing the author to define a "short title" to be used in page headers.
\title[Title for top of each page]{Regular title text}

\author{Author Name}
\affiliation{%
  \institution{Institution name}
  \streetaddress{Institution address}
  \city{institution city}
  \state{State}
  \postcode{99999}
}
\email{Author Email}

\author{Author Name}
\affiliation{%
  \institution{Institution name}
  \streetaddress{Institution address}
  \city{institution city}
  \state{State}
  \postcode{99999}
}

\email{Author Email}
\author{Author Name}
\affiliation{%
  \institution{Institution name}
  \streetaddress{Institution address}
  \city{institution city}
  \state{State}
  \postcode{99999}
}

\email{Author Email}
\author{Author Name}
\affiliation{%
  \institution{Institution name}
  \streetaddress{Institution address}
  \city{institution city}
  \state{State}
  \postcode{99999}
}
\email{Author Email}

\author{Author Name}
\affiliation{%
  \institution{Institution name}
  \streetaddress{Institution address}
  \city{institution city}
  \state{State}
  \postcode{99999}
}
\email{Author Email}
%
% By default, the full list of authors will be used in the page headers. Often, this list is too long, and will overlap
% other information printed in the page headers. This command allows the author to define a more concise list
% of authors' names for this purpose.
\renewcommand{\shortauthors}{Authors, et al.}

%
% The abstract is a short summary of the work to be presented in the article.
\begin{abstract}
Abstract text. 
\end{abstract}

\begin{CCSXML}
<ccs2012>
<concept>
<concept_id>10003456.10003457.10003527.10003528</concept_id>
<concept_desc>Social and professional topics~Computational thinking</concept_desc>
<concept_significance>500</concept_significance>
</concept>
<concept>
<concept_id>10003456.10003457.10003527.10003531.10003533</concept_id>
<concept_desc>Social and professional topics~Computer science education</concept_desc>
<concept_significance>500</concept_significance>
</concept>
<concept>
<concept_id>10003456.10003457.10003527.10003541</concept_id>
<concept_desc>Social and professional topics~K-12 education</concept_desc>
<concept_significance>500</concept_significance>
</concept>
</ccs2012>
\end{CCSXML}


%
% Keywords. 
\keywords{keywords}
    
\maketitle

\section{Introduction} 
Introduction text.

Example reference and another one with two references. 

\section{Section header text} 
\cite{ch00}
\cite{ch01}
\cite{ch02}
\cite{ch03}
\cite{ch04}
\cite{ch05}
\cite{ch06}
\cite{ch07}
\cite{ch08}
\cite{ch09}
\cite{ch10}
\cite{ch11}
\cite{ch12}
\cite{ch13}
\cite{ch14}
\cite{ch15}
\cite{ch16}
\cite{ch17}
\cite{ch18}
\cite{ch19}
\cite{ch20}
\cite{ch21}
\cite{ch22}
\cite{ch23, ch24, ch25}

\subsection{Example Subsection header} 

\cite{ch26}
\cite{ch27}
\cite{ch28}
\cite{ch29}
\cite{ch30}
\cite{ch31}



\subsubsection{Example sub-sub section header} 

\begin{itemize}
    \item Example bullet,
    \item another bullet
\end{itemize}



\begin{acks}
Acknowledgement text.
\end{acks}

%
% The next two lines define the bibliography style to be used, and the bibliography file.
\bibliographystyle{ACM-Reference-Format}
\bibliography{sample-base}

\end{document}
